\documentclass{article}

\usepackage[margin=1in]{geometry}
\usepackage{amsmath}
\usepackage{hyperref}
\usepackage{graphicx}
\usepackage{enumitem}
\usepackage{xcolor}
\usepackage{amssymb}
\usepackage{microtype}
\usepackage{titlesec}
\usepackage{tcolorbox}
\usepackage{booktabs}
\usepackage[most]{tcolorbox}

\titleformat{\section}{\large\bfseries\color{blue}}{\thesection}{1em}{}
\titleformat{\subsection}{\normalsize\bfseries\color{teal}}{\thesubsection}{1em}{}

\definecolor{lightblue}{RGB}{173, 216, 230}
\definecolor{lightgrey}{RGB}{240, 240, 240}

\hypersetup{
    colorlinks=true,
    linkcolor=blue,
    filecolor=magenta,
    urlcolor=cyan,
}

\title{\textbf{Distributed Systems: An Introduction}}
\author{Aayush Adhikari}
\date{\today}

\begin{document}
\maketitle

\section{Introduction}
A \textbf{distributed system} is a collection of independent computers that work together to achieve a common goal. The computers communicate with each other through message passing or events.

\textbf{Distributed computing} is a type of computing that takes place on distributed systems. It involves building and designing computing models for distributed systems and developing algorithms to solve problems related to such systems.

\section{Characteristics of Distributed Systems}
\begin{itemize}
    \item No shared clock: Each computer has its own clock, making coordination difficult.
    \item No shared memory: Each computer has its own memory, preventing direct memory access.
    \item Concurrency: Multiple tasks can be executed simultaneously.
    \item Heterogeneity: Computers in the system can be different from each other.
    \item Loose coupling: Computers are loosely coupled and not tightly integrated.
\end{itemize}

\section{Basic Concepts of Distributed Computing}
\begin{description}[itemsep=5pt, parsep=0pt]
    \item[Nodes:] The individual computers in a distributed system.
    \item[Resources:] Assets that can be accessed remotely by nodes (e.g., data, storage).
    \item[Distribution Transparency:] Hiding the complexities of the system from users.
    \item[Middleware:] Software layer facilitating communication and management between nodes.
    \item[Concurrency:] Ability of the system to execute multiple tasks simultaneously.
    \item[Coordination \& Synchronization:] Techniques to ensure cooperation between tasks.
    \item[Architectural Models:] Define how nodes are organized (e.g., client-server, peer-to-peer).
    \item[Global State:] The collective state of all nodes in the system.
\end{description}

\section{Advantages of Distributed Systems}
\begin{itemize}
    \item Reliability: Can continue operating even if one or more computers fail.
    \item Scalability: Can easily scale by adding more computers.
    \item Resource sharing: Resources can be shared between computers.
    \item Increased performance: Tasks can be distributed among multiple computers.
\end{itemize}

\section{Disadvantages of Distributed Systems}
\begin{itemize}
    \item Failure detection: Difficult to detect failures in the system.
    \item Redundancy: Multiple computers may have copies of the same data.
    \item Consistency: Difficult to ensure all computers have the same data.
    \item Performance bottlenecks: Can occur due to slow network or resource competition.
\end{itemize}

\end{document}
